\documentclass[12pt]{article}
\usepackage{geometry}
\geometry{a4paper, margin=1in}
\usepackage{fancyhdr}
\usepackage{hyperref}
\usepackage{graphicx}
\usepackage{longtable}

\title{Software Requirements Specification (SRS) \\ \textbf{El Mundo Bites}}
\author{BiteBytes Group}
\date{9 November 2024}

\begin{document}
\maketitle

\thispagestyle{empty}
\vspace{2cm}

\begin{center}
    \includegraphics[width=0.4\textwidth]{group_logo.png} % Add your logo here
\end{center}

\newpage
\pagenumbering{Roman}
\tableofcontents
\newpage

\section*{Revision History}
\begin{longtable}{|c|c|c|l|}
\hline
\textbf{Version} & \textbf{Date}       & \textbf{Author}  & \textbf{Description} \\ \hline
1.0              & 9 November 2024 & BiteBytes Group & document draft \\ \hline
\end{longtable}

\newpage
\pagenumbering{arabic}

\section{Introduction}

\subsection{Purpose}
The purpose of this Software Requirements Specification (SRS) document is to outline the requirements for the mobile application, \textbf{El Mundo Bites}. This app is designed to provide users with a unique cultural experience by discovering and interacting with recipes from various global cuisines. This document details functional and non-functional requirements to guide the development and testing of this application.

\subsection{Scope}
El Mundo Bites is a mobile application that allows users to:
\begin{itemize}
    \item Search for recipes by ingredients, cuisine type, or country.
    \item Save favorite recipes for easy access.
    \item Rate and review recipes.
    \item Receive personalized recipe recommendations based on past interactions.
\end{itemize}
The app aims to engage food enthusiasts, students, chefs, and travelers who want to explore international cuisine in an accessible, user-friendly format.

\subsection{Intended Audience}
This document is intended for:
\begin{itemize}
    \item \textbf{Developers} to implement the features and functionality described.
    \item \textbf{Project Managers} to monitor project progress and requirements.
    \item \textbf{Testers} to validate features according to the specifications.
    \item \textbf{End Users} to gain an understanding of the app’s intended capabilities.
\end{itemize}

\subsection{References}
\begin{itemize}
    \item IEEE Std 830-1998, IEEE Recommended Practice for Software Requirements Specifications
    \item Project Proposal Document
    \item Requirements Determination and User Interface Design materials
\end{itemize}

\newpage
\section{Overall Description}

\subsection{Product Perspective}
El Mundo Bites offers a platform for culinary exploration by connecting users with recipes from various cultures. Unlike general recipe apps, El Mundo Bites emphasizes a cultural approach, enabling users to learn about the origins of recipes and connect with a diverse community of culinary enthusiasts.

\subsection{User Classes and Characteristics}
\begin{itemize}
    \item \textbf{Students}: Users who seek affordable, easy-to-prepare meals.
    \item \textbf{Professional Chefs}: Users with advanced culinary skills who provide expert reviews.
    \item \textbf{Food Enthusiasts}: Hobbyist cooks interested in exploring and sharing recipes.
    \item \textbf{Travelers}: Users curious about recipes associated with their travel destinations.
\end{itemize}

\subsection{Operating Environment}
The application is designed for Android and iOS platforms and requires internet connectivity for data retrieval from external APIs. The app should work optimally on a range of screen sizes and devices, ensuring a responsive design.

\subsection{Assumptions and Dependencies}
\begin{itemize}
    \item The app will require a stable internet connection for retrieving recipes.
    \item Recipe data will be sourced from third-party APIs, which are assumed to be reliable and accessible.
    \item User data will be stored securely on a cloud-based platform.
\end{itemize}

\newpage
\section{System Features (Functional Requirements)}

\subsection{Recipe Search}
\textbf{Description:} Allows users to search for recipes based on ingredients, cuisine, or country.

\textbf{Stimulus/Response Sequences:}
\begin{itemize}
    \item \textbf{Stimulus:} The user enters search criteria in the search bar.
    \item \textbf{Response:} The system displays a list of recipes matching the specified criteria.
\end{itemize}

\textbf{Functional Requirements:}
\begin{itemize}
    \item FR-3.1.1: Provide a search bar on the main screen where users can input keywords.
    \item FR-3.1.2: Allow search filters for cuisine type, country, and ingredients.
    \item FR-3.1.3: Display a paginated list of results, with 20 results per page.
\end{itemize}

\subsection{Favorites Management}
\textbf{Description:} Enables users to save and organize their favorite recipes for future access.

\textbf{Stimulus/Response Sequences:}
\begin{itemize}
    \item \textbf{Stimulus:} The user selects the “Save” option on a recipe.
    \item \textbf{Response:} The recipe is added to the user's favorites section.
\end{itemize}

\textbf{Functional Requirements:}
\begin{itemize}
    \item FR-3.2.1: Provide a “Save to Favorites” icon on each recipe.
    \item FR-3.2.2: Display saved recipes under the “Favorites” menu.
    \item FR-3.2.3: Allow users to remove recipes from their favorites.
\end{itemize}

\subsection{User Rating and Review System}
\textbf{Description:} Users can leave ratings and reviews for recipes, providing feedback for other users.

\textbf{Stimulus/Response Sequences:}
\begin{itemize}
    \item \textbf{Stimulus:} The user submits a rating and a written review.
    \item \textbf{Response:} The system displays the user’s rating and review on the recipe page.
\end{itemize}

\textbf{Functional Requirements:}
\begin{itemize}
    \item FR-3.3.1: Include a 5-star rating system for each recipe.
    \item FR-3.3.2: Allow users to submit written reviews.
    \item FR-3.3.3: Display the average rating based on user submissions.
\end{itemize}

\subsection{Recipe Recommendation System}
\textbf{Description:} Recommends recipes based on user activity and preferences.

\textbf{Stimulus/Response Sequences:}
\begin{itemize}
    \item \textbf{Stimulus:} The user opens the app’s home screen.
    \item \textbf{Response:} The system displays a list of personalized recipe recommendations.
\end{itemize}

\textbf{Functional Requirements:}
\begin{itemize}
    \item FR-3.4.1: Provide personalized recommendations on the home screen.
    \item FR-3.4.2: Update recommendations based on new user data.
\end{itemize}

\subsection{Error Handling and Edge Cases}
\textbf{Description:} Defines how the system should respond to unexpected inputs, network issues, and empty states.

\textbf{Functional Requirements:}
\begin{itemize}
    \item \textbf{FR-3.5.1:} Display a “No Results Found” message when no recipes match the search criteria.
    \item \textbf{FR-3.5.2:} If the network connection is lost, display a “Connection Lost” message with a “Retry” option.
    \item \textbf{FR-3.5.3:} Validate all inputs to ensure they meet expected formats; provide specific error messages if inputs are invalid.
\end{itemize}

\newpage
\section{External Interface Requirements}

\subsection{User Interfaces}
\begin{itemize}
    \item \textbf{Home Screen}: Displays recipe recommendations and search functionality.
    \item \textbf{Recipe Detail Screen}: Shows details, ratings, reviews, and the “Save” option.
    \item \textbf{Favorites Screen}: Lists saved recipes for easy access.
    \item \textbf{Review Screen}: Allows users to submit ratings and reviews.
\end{itemize}

\subsection{Hardware Interfaces}
The application will be compatible with Android and iOS devices, supporting a range of screen sizes.

\subsection{Software Interfaces}
\begin{itemize}
    \item \textbf{APIs}: Integrates with third-party recipe databases for recipe content.
    \item \textbf{Database}: Stores user preferences, saved recipes, and profile data on cloud servers.
\end{itemize}

\subsection{Communication Interfaces}
The app requires an active internet connection to access external APIs for recipe data retrieval.

\subsection{Error Messages and Notifications}
\paragraph{User Interfaces:}
\begin{itemize}
    \item Display clear error messages (“No Results Found”) in place of search results when applicable.
    \item For critical errors (failed data retrieval), display a pop-up notification with an option to retry.
\end{itemize}

\newpage
\section{Nonfunctional Requirements}

\subsection{Performance}
\begin{itemize}
    \item The system shall load search results within 3 seconds on a stable internet connection.
    \item The app shall support up to 10,000 concurrent users without performance degradation.
\end{itemize}

\subsection{Usability}
\begin{itemize}
    \item The app shall provide an intuitive, user-friendly interface with a minimal learning curve.
    \item The app shall comply with accessibility standards (e.g., WCAG 2.1) for screen readers.
\end{itemize}

\subsection{Security}
\begin{itemize}
    \item User data shall be encrypted during both storage and transmission.
    \item User sessions shall timeout after 24 hours of inactivity for security purposes.
\end{itemize}

\subsection{Data Security and Compliance}
\textbf{Description:} Ensures that user data is handled in accordance with data protection standards.

\textbf{Nonfunctional Requirements:}
\begin{itemize}
    \item \textbf{NFR-5.4.1:} Ensure compliance with data protection regulations, such as GDPR, for users within applicable regions.
    \item \textbf{NFR-5.4.2:} Implement an option for users to request data deletion in compliance with data privacy rights.
\end{itemize}

\subsection{Scalability and Future Expansion}
\textbf{Description:} Provides a foundation for future feature expansion, such as integrating social features or additional user recommendations.

\textbf{Nonfunctional Requirements:}
\begin{itemize}
    \item \textbf{NFR-5.5.1:} Ensure the system architecture allows for the integration of social sharing options without major refactoring.
    \item \textbf{NFR-5.5.2:} Allow the recommendation system to incorporate AI-based suggestions or dietary preference filters in future releases.
\end{itemize}

\newpage
\section{Appendices}

\subsection{Glossary}
\begin{itemize}
    \item \textbf{API}: Application Programming Interface.
    \item \textbf{User Profile}: Stores personal settings and saved recipes for each user.
    \item \textbf{Recipe Database}: A third-party resource for recipe information and metadata.
\end{itemize}

\subsection{Data Flow Diagrams}
\begin{itemize}
    \item \textbf{User Authentication Flow}: Diagrams the process for logging in and session management.
    \item \textbf{Recipe Retrieval Flow}: Illustrates data retrieval from the recipe database to the app interface.
\end{itemize}

\subsection{Use Case Diagrams}
\textbf{Use Case: Recipe Search}
\begin{itemize}
    \item \textbf{Actors:} User
    \item \textbf{Description:} The user enters keywords in the search bar and selects search criteria (ingredients, cuisine, country).
    \item \textbf{Preconditions:} The user is logged in, and a stable network connection is available.
    \item \textbf{Postconditions:} A list of matching recipes is displayed, or a “No Results Found” message if no matches are found.
\end{itemize}

\subsection{Sequence Diagrams}
\paragraph{Sequence Diagram: Saving a Recipe to Favorites}
\begin{itemize}
    \item \textbf{Description:} This diagram illustrates the interaction between the user interface and backend systems when a user saves a recipe.
\end{itemize}

\end{document}
